\documentclass{article}
\usepackage[utf8]{inputenc}
\usepackage{geometry}
\usepackage{hyperref}
\usepackage{listings}
\usepackage{xcolor}
\usepackage{tabulary}
\usepackage{booktabs}
\usepackage{enumitem}

\geometry{a4paper, margin=1in}

\hypersetup{
    colorlinks=true,
    linkcolor=blue,
    filecolor=magenta,
    urlcolor=cyan,
}

\lstset{
    basicstyle=\ttfamily\small,
    breaklines=true,
    commentstyle=\color{gray},
    keywordstyle=\color{blue},
    stringstyle=\color{red},
    numbers=left,
    numberstyle=\tiny\color{gray},
    numbersep=5pt,
    frame=single,
    framesep=5pt,
    framexleftmargin=15pt,
    tabsize=4,
    captionpos=b
}

\title{StatApp Command Line Interface Documentation}
\author{StatApp Team}
\date{\today}

\begin{document}

\maketitle
\tableofcontents
\newpage

\section{Introduction}
This document provides comprehensive documentation for the StatApp command line interface. The StatApp CLI is a tool for medical segmentation that handles inter-rater uncertainty and variability. It is built on top of nnU-Net, a framework for out-of-the-box image segmentation.

The CLI provides commands for:
\begin{itemize}
    \item Preparing datasets
    \item Training models
    \item Running predictions
    \item Ensembling multiple models
    \item Computing metrics
    \item Managing artifacts in S3 storage
\end{itemize}

\section{Command Overview}

\subsection{about}
\subsubsection{Description}
Displays information about the project, including its purpose and contributors.

\subsubsection{Usage}
\begin{lstlisting}
statapp about
\end{lstlisting}

\subsubsection{Parameters}
None

\subsection{upload}
\subsubsection{Description}
The upload command provides functionality to upload local directories to S3 storage. It includes three subcommands:

\paragraph{upload\_data}
Uploads a local directory to the S3 data folder defined in the .env file.

\paragraph{upload\_model\_artifacts}
Uploads a local directory to the S3 artifacts/model folder defined in the .env file.

\paragraph{upload\_preprocessing\_artifacts}
Uploads a local directory to the S3 artifacts/preprocessing folder defined in the .env file.

\subsubsection{Usage}
\begin{lstlisting}
statapp upload upload-data <directory> [--verbose]
statapp upload upload-model-artifacts <directory> <modelfolder> [--verbose]
statapp upload upload-preprocessing-artifacts <directory> <preprocessingfolder> [--verbose]
\end{lstlisting}

\subsubsection{Parameters}
\begin{tabulary}{\linewidth}{LLLL}
\toprule
\textbf{Subcommand} & \textbf{Parameter} & \textbf{Type} & \textbf{Description} \\
\midrule
upload-data & directory & string & Local directory path to upload \\
upload-data & --verbose, -v & flag & Enable verbose output \\
\midrule
upload-model-artifacts & directory & string & Local directory path to upload \\
upload-model-artifacts & modelfolder & string & Subfolder name of the model \\
upload-model-artifacts & --verbose, -v & flag & Enable verbose output \\
\midrule
upload-preprocessing-artifacts & directory & string & Local directory path to upload \\
upload-preprocessing-artifacts & preprocessingfolder & string & Subfolder name of the preprocessing \\
upload-preprocessing-artifacts & --verbose, -v & flag & Enable verbose output \\
\bottomrule
\end{tabulary}

\subsection{empty-artifacts}
\subsubsection{Description}
Removes all files and folders from the S3 artifacts folder defined in the .env file. This operation cannot be undone, so use with caution.

\subsubsection{Usage}
\begin{lstlisting}
statapp empty-artifacts [--verbose] [--confirm]
\end{lstlisting}

\subsubsection{Parameters}
\begin{tabulary}{\linewidth}{LLL}
\toprule
\textbf{Parameter} & \textbf{Type} & \textbf{Description} \\
\midrule
--verbose, -v & flag & Enable verbose output \\
--confirm, -c & flag & Confirm deletion without prompting \\
\bottomrule
\end{tabulary}

\subsection{empty-data}
\subsubsection{Description}
Removes all files and folders from the S3 data folder defined in the .env file. This operation cannot be undone, so use with caution.

\subsubsection{Usage}
\begin{lstlisting}
statapp empty-data [--verbose] [--confirm]
\end{lstlisting}

\subsubsection{Parameters}
\begin{tabulary}{\linewidth}{LLL}
\toprule
\textbf{Parameter} & \textbf{Type} & \textbf{Description} \\
\midrule
--verbose, -v & flag & Enable verbose output \\
--confirm, -c & flag & Confirm deletion without prompting \\
\bottomrule
\end{tabulary}

\subsection{prepare}
\subsubsection{Description}
The prepare command provides functionality for preparing datasets for analysis. It includes three subcommands:

\paragraph{download-dataset}
Downloads a dataset for analysis without running preprocessing.

\paragraph{download-preprocessing}
Downloads preprocessing artifacts for a dataset.

\paragraph{prepare}
Prepares a dataset for analysis by downloading data and running preprocessing.

\subsubsection{Usage}
\begin{lstlisting}
statapp prepare download-dataset <annotator> <patients> [--verbose]
statapp prepare download-preprocessing <annotator> <patients> [--verbose]
statapp prepare prepare <annotator> <patients> [--skip] [--num-processes-fingerprint <num>] [--num-processes <num>] [--verbose]
\end{lstlisting}

\subsubsection{Parameters}
\begin{tabulary}{\linewidth}{LLLL}
\toprule
\textbf{Subcommand} & \textbf{Parameter} & \textbf{Type} & \textbf{Description} \\
\midrule
download-dataset & annotator & string & Annotator (1/2/3) \\
download-dataset & patients & list/string & List of patient numbers (e.g., 001 034) or 'all', 'train', 'validation', 'test' \\
download-dataset & --verbose, -v & flag & Enable verbose logging \\
\midrule
download-preprocessing & annotator & string & Annotator (1/2/3) \\
download-preprocessing & patients & list/string & List of patient numbers (e.g., 001 034) or 'all', 'train', 'validation', 'test' \\
download-preprocessing & --verbose, -v & flag & Enable verbose logging \\
\midrule
prepare & annotator & string & Annotator (1/2/3) \\
prepare & patients & list/string & List of patient numbers (e.g., 001 034) or 'all', 'train', 'validation', 'test' \\
prepare & --skip & flag & Skip download and only run preprocessing \\
prepare & --num-processes-fingerprint, -npfp & integer & Number of processes to use for fingerprint extraction (default: 2) \\
prepare & --num-processes, -np & integer & Number of processes to use for preprocessing (default: 2) \\
prepare & --verbose, -v & flag & Enable verbose logging \\
\bottomrule
\end{tabulary}

\subsection{train}
\subsubsection{Description}
Runs nnUNet training. The dataset must be prepared with the prepare command beforehand.

\subsubsection{Usage}
\begin{lstlisting}
statapp train [<seed>] [--fold <fold>] [--patients <patients>] [--annotator <annotator>] [--verbose]
\end{lstlisting}

\subsubsection{Parameters}
\begin{tabulary}{\linewidth}{LLL}
\toprule
\textbf{Parameter} & \textbf{Type} & \textbf{Description} \\
\midrule
seed & integer & Set random seed for reproducibility \\
--fold, -f & string & Fold to use for training. Can be 'all' to use all folds, or a specific fold number (0-4) \\
--patients, -p & list/string & List of patient numbers (e.g., 001 034) or 'all', 'train', 'validation', 'test' \\
--annotator, -a & string & Annotator (1/2/3) \\
--verbose, -v & flag & Enable verbose logging \\
\bottomrule
\end{tabulary}

\subsection{run}
\subsubsection{Description}
Runs the complete pipeline: prepare data, train model, and upload artifacts. This command combines the functionality of prepare, train, and upload\_artifacts commands.

\subsubsection{Usage}
\begin{lstlisting}
statapp run <annotator> <seed> <patients> [--fold <fold>] [--skip] [--num-processes-fingerprint <num>] [--num-processes <num>] [--verbose]
\end{lstlisting}

\subsubsection{Parameters}
\begin{tabulary}{\linewidth}{LLL}
\toprule
\textbf{Parameter} & \textbf{Type} & \textbf{Description} \\
\midrule
annotator & string & Annotator (1/2/3) \\
seed & integer & Set random seed for reproducibility \\
patients & list/string & List of patient numbers (e.g., 001 034) or 'all', 'train', 'validation', 'test' \\
--fold, -f & string & Fold to use for training. Can be 'all' to use all folds, or a specific fold number (0-4) \\
--skip & flag & Skip download and only run preprocessing \\
--num-processes-fingerprint, -npfp & integer & Number of processes to use for fingerprint extraction (default: 2) \\
--num-processes, -np & integer & Number of processes to use for preprocessing (default: 2) \\
--verbose, -v & flag & Enable verbose logging \\
\bottomrule
\end{tabulary}

\subsection{predict}
\subsubsection{Description}
Predicts segmentation for patients using specified models. Downloads patient images and model checkpoints, runs prediction for each model, and uploads results to S3.

\subsubsection{Usage}
\begin{lstlisting}
statapp predict <patients> [--models <models>] [--jobs <num>] [--verbose]
\end{lstlisting}

\subsubsection{Parameters}
\begin{tabulary}{\linewidth}{LLL}
\toprule
\textbf{Parameter} & \textbf{Type} & \textbf{Description} \\
\midrule
patients & list/string & List of patient numbers (e.g., 001 034) or 'all', 'train', 'validation', 'test' \\
--models, -m & list/string & List of models to use for prediction (e.g., anno1\_init112233\_foldall) or 'all' \\
--jobs, -j & integer & Number of processes to run (default: 10) \\
--verbose, -v & flag & Enable verbose logging \\
\bottomrule
\end{tabulary}

\subsection{ensemble}
\subsubsection{Description}
The ensemble command provides functionality for ensembling predictions from multiple models. It includes three subcommands:

\paragraph{dl-ensemble}
Downloads predictions from multiple models.

\paragraph{run-ensemble}
Runs ensemble\_folders from nnUNet on the downloaded model predictions.

\paragraph{ensemble}
Combines the functionality of both dl-ensemble and run-ensemble commands.

\subsubsection{Usage}
\begin{lstlisting}
statapp ensemble dl-ensemble <patients> [--models <models>] [--jobs <num>] [--verbose]
statapp ensemble run-ensemble <patients> [--verbose] [--jobs <num>]
statapp ensemble ensemble <patients> [--models <models>] [--jobs <num>] [--verbose]
\end{lstlisting}

\subsubsection{Parameters}
\begin{tabulary}{\linewidth}{LLLL}
\toprule
\textbf{Subcommand} & \textbf{Parameter} & \textbf{Type} & \textbf{Description} \\
\midrule
dl-ensemble & patients & list/string & List of patient numbers (e.g., 001 034) or 'all', 'train', 'validation', 'test' \\
dl-ensemble & --models, -m & list/string & List of models to use for ensembling (e.g., anno1\_init112233\_foldall) or 'all' \\
dl-ensemble & --jobs, -j & integer & Number of processes to run (default: 10) \\
dl-ensemble & --verbose, -v & flag & Enable verbose logging \\
\midrule
run-ensemble & patients & list/string & List of patient numbers (e.g., 001 034) \\
run-ensemble & --jobs, -j & integer & Number of processes to run (default: 10) \\
run-ensemble & --verbose, -v & flag & Enable verbose logging \\
\midrule
ensemble & patients & list/string & List of patient numbers (e.g., 001 034) or 'all', 'train', 'validation', 'test' \\
ensemble & --models, -m & list/string & List of models to use for ensembling (e.g., anno1\_init112233\_foldall) or 'all' \\
ensemble & --jobs, -j & integer & Number of processes to run (default: 10) \\
ensemble & --verbose, -v & flag & Enable verbose logging \\
\bottomrule
\end{tabulary}

\subsection{metrics}
\subsubsection{Description}
The metrics command provides functionality for computing and downloading metrics for model predictions. It includes two subcommands:

\paragraph{compute-metrics}
Computes metrics for model predictions on patient data.

\paragraph{dl-metrics}
Downloads all metrics files from S3, merges them, and saves to the working directory.

\subsubsection{Usage}
\begin{lstlisting}
statapp metrics compute-metrics <patients> [--models <models>] [--verbose]
statapp metrics dl-metrics [--output <output_name>] [--verbose]
\end{lstlisting}

\subsubsection{Parameters}
\begin{tabulary}{\linewidth}{LLLL}
\toprule
\textbf{Subcommand} & \textbf{Parameter} & \textbf{Type} & \textbf{Description} \\
\midrule
compute-metrics & patients & list/string & List of patient numbers (e.g., 075 034) or 'all', 'train', 'validation', 'test' \\
compute-metrics & --models, -m & list/string & List of models to use for metrics computation (e.g., anno1\_init112233\_foldall) or 'all' \\
compute-metrics & --verbose, -v & flag & Enable verbose logging \\
\midrule
dl-metrics & --output, -o & string & Name of the output CSV file (default: metrics.csv) \\
dl-metrics & --verbose, -v & flag & Enable verbose logging \\
\bottomrule
\end{tabulary}

\section{Examples}

\subsection{Preparing a Dataset}
\begin{lstlisting}
# Download a dataset for annotator 1 with all patients
statapp prepare download-dataset 1 all --verbose

# Download preprocessing artifacts for annotator 1 with training patients
statapp prepare download-preprocessing 1 train --verbose

# Prepare a dataset for annotator 1 with validation patients
statapp prepare prepare 1 validation --num-processes 4 --verbose
\end{lstlisting}

\subsection{Training a Model}
\begin{lstlisting}
# Train a model with annotator 1, seed 42, and all folds
statapp train 42 --annotator 1 --fold all --patients train --verbose

# Run the complete pipeline for annotator 1, seed 42, and validation patients
statapp run 1 42 validation --fold all --num-processes 4 --verbose
\end{lstlisting}

\subsection{Running Predictions}
\begin{lstlisting}
# Predict segmentation for test patients using all models
statapp predict test --models all --jobs 8 --verbose

# Predict segmentation for specific patients using a specific model
statapp predict 001 034 --models anno1_init42_foldall --jobs 4 --verbose
\end{lstlisting}

\subsection{Ensembling Models}
\begin{lstlisting}
# Download predictions for test patients from all models
statapp ensemble dl-ensemble test --models all --jobs 8 --verbose

# Run ensemble for specific patients
statapp ensemble run-ensemble 001 034 --jobs 4 --verbose

# Run the complete ensemble pipeline for test patients
statapp ensemble ensemble test --models all --jobs 8 --verbose
\end{lstlisting}

\subsection{Computing Metrics}
\begin{lstlisting}
# Compute metrics for test patients using all models
statapp metrics compute-metrics test --models all --verbose

# Download and merge all metrics files
statapp metrics dl-metrics --output combined_metrics.csv --verbose
\end{lstlisting}

\subsection{Managing S3 Storage}
\begin{lstlisting}
# Upload data to S3
statapp upload upload-data ./my_data --verbose

# Upload model artifacts to S3
statapp upload upload-model-artifacts ./my_model anno1_init42_foldall --verbose

# Empty the artifacts directory (with confirmation)
statapp empty-artifacts --verbose

# Empty the data directory (without confirmation)
statapp empty-data --confirm --verbose
\end{lstlisting}

\section{Environment Variables}
The StatApp CLI uses several environment variables for configuration. These should be defined in a .env file in the project root directory.

\begin{tabulary}{\linewidth}{LL}
\toprule
\textbf{Variable} & \textbf{Description} \\
\midrule
S3\_BUCKET & The S3 bucket name \\
S3\_DATA\_DIR & The directory in S3 for storing data \\
S3\_ARTIFACTS\_DIR & The directory in S3 for storing artifacts \\
S3\_OUTPUT\_DIR & The directory in S3 for storing output \\
S3\_METRICS\_DIR & The directory in S3 for storing metrics \\
S3\_MODEL\_ARTIFACTS\_SUBDIR & The subdirectory for model artifacts (default: models) \\
S3\_PROPROCESSING\_ARTIFACTS\_SUBDIR & The subdirectory for preprocessing artifacts (default: preprocessing) \\
SEED & Random seed for reproducibility \\
\bottomrule
\end{tabulary}

\end{document}